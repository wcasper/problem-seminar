% !TEX TS-program = pdflatex
% !TEX encoding = UTF-8 Unicode

% This file is a template using the "beamer" package to create slides for a talk or presentation
% - Giving a talk on some subject.
% - The talk is between 15min and 45min long.
% - Style is ornate.

% MODIFIED by Jonathan Kew, 2008-07-06
% The header comments and encoding in this file were modified for inclusion with TeXworks.
% The content is otherwise unchanged from the original distributed with the beamer package.

\documentclass{beamer}


% Copyright 2004 by Till Tantau <tantau@users.sourceforge.net>.
%
% In principle, this file can be redistributed and/or modified under
% the terms of the GNU Public License, version 2.
%
% However, this file is supposed to be a template to be modified
% for your own needs. For this reason, if you use this file as a
% template and not specifically distribute it as part of a another
% package/program, I grant the extra permission to freely copy and
% modify this file as you see fit and even to delete this copyright
% notice. 

\addtobeamertemplate{navigation symbols}{}{%
    \usebeamerfont{footline}%
    \usebeamercolor[fg]{footline}%
    \hspace{1em}%
    \insertframenumber/\inserttotalframenumber
}

\mode<presentation>
{
  \usetheme{Warsaw}
  % or ...

  \setbeamercovered{transparent}
  % or whatever (possibly just delete it)
}

\usepackage{hyperref}

\usepackage[english]{babel}
% or whatever

\usepackage[utf8]{inputenc}
% or whatever

\usepackage{times}
\usepackage[T1]{fontenc}
% Or whatever. Note that the encoding and the font should match. If T1
% does not look nice, try deleting the line with the fontenc.



\title{Recursion Problems}

\author[W.R. Casper] % (optional, use only with lots of authors)
{W.R. Casper\\\vspace{0.2in}
 {\tiny Problem Solving Seminar}}
% - Use the \inst{?} command only if the authors have different
%   affiliation.

\institute[California State University Fullerton] % (optional, but mostly needed)
{
  Department of Mathematics\\
  California State University Fullerton}
% - Use the \inst command only if there are several affiliations.
% - Keep it simple, no one is interested in your street address.

\subject{Talks}
% This is only inserted into the PDF information catalog. Can be left
% out. 



% If you have a file called "university-logo-filename.xxx", where xxx
% is a graphic format that can be processed by latex or pdflatex,
% resp., then you can add a logo as follows:

% \pgfdeclareimage[height=0.5cm]{university-logo}{university-logo-filename}
% \logo{\pgfuseimage{university-logo}}



% Delete this, if you do not want the table of contents to pop up at
% the beginning of each subsection:
\AtBeginSubsection[]
{
  \begin{frame}<beamer>{Outline}
    \tableofcontents[currentsection,currentsubsection]
  \end{frame}
}


% If you wish to uncover everything in a step-wise fashion, uncomment
% the following command: 

%\beamerdefaultoverlayspecification{<+->}


\begin{document}

\begin{frame}
  \titlepage
\end{frame}

%\begin{frame}{Outline}
  %\tableofcontents
  % You might wish to add the option [pausesections]
%\end{frame}

% Since this a solution template for a generic talk, very little can
% be said about how it should be structured. However, the talk length
% of between 15min and 45min and the theme suggest that you stick to
% the following rules:  

% - Exactly two or three sections (other than the summary).
% - At *most* three subsections per section.
% - Talk about 30s to 2min per frame. So there should be between about
%   15 and 30 frames, all told.

\begin{frame}{Recursive definition}
\begin{center}
\includegraphics[height=1in]{fig/xkcd-recursion}
\end{center}
Recursive definitions involves two things:
\begin{itemize}
\item a base case -- where the thing is defined explicitly
\item a recursive step -- a rule which relates the next case to previous cases
\end{itemize}
\end{frame}

\begin{frame}{Putnam 1990 A1}
Let $T_0 = 2, T_1 = 3, T_2 = 6$, and for $n \geq 3$,
$$T_n = (n + 4)T_{n-1} - 4nT_{n-2} + (4n - 8)T_{n-3}.$$
The first few terms are
$$2, 3, 6, 14, 40, 152, 784, 5168, 40576, 363392$$
Find, with proof, a formula for $T_n$ of the form $T_n = A_n+B_n$, where $(A_n)$ and $(B_n)$ are well-known sequences.
\end{frame}
\begin{frame}{Putnam 1990 A1}
Let $T_0 = 2, T_1 = 3, T_2 = 6$, and for $n \geq 3$,
$$T_n = (n + 4)T_{n-1} - 4nT_{n-2} + (4n - 8)T_{n-3}.$$
The first few terms are
$$2, 3, 6, 14, 40, 152, 784, 5168, 40576, 363392$$
Find, with proof, a formula for $T_n$ of the form $T_n = A_n+B_n$, where $(A_n)$ and $(B_n)$ are well-known sequences.\\
{\color{red}Hint: compare to the first few terms of the sequence $n!$
$$1, 1, 2, 6, 24, 120, 720, 5040, 40320, 362880$$}
\end{frame}


\begin{frame}{Fibonacci Sequence}
\begin{itemize}
\item Base case:
$$F_1 = 1,\ \ F_2 = 1,$$
\item Recursive step:
$$F_{n+1} = F_n + F_{n-1},\ \ n\geq 2$$
\end{itemize}
$$F_3 = 2,\ \ F_4 = 3,\ \ F_5 = 5,\ \ F_6 = 8,\dots...$$ 
\begin{center}
\includegraphics[height=1in]{fig/fibonacci}
\end{center}
\end{frame}

\begin{frame}{A closed form}
Show that
$$F_n = \frac{\varphi^n-(-\varphi)^{-n}}{\varphi+\varphi^{-1}},$$
where here $\varphi$ is the \textbf{golden ratio}
$$\varphi = \frac{1+\sqrt{5}}{2}.$$
\end{frame}

\begin{frame}{Practice again}
Suppose that $u_0 = 0$, $u_1 = 1$ and 
$$u_{n+2} = 4u_{n+1}-4u_n.$$
Determine the value of $u_{16}$.
\end{frame}

\begin{frame}{Generalization}
Suppose that $u_0, u_1$, and $u_2$ are three real numbers.
Define recursively
$$u_{n+2} = \frac{u_{n+1}+u_n+u_{n-1}}{3}.$$
Determine the limit of $u_{n}$.
\end{frame}

\begin{frame}{Putnam 2015 A2}
Suppose that $a_0 = 1, a_1 = 2$, and
$$a_{n} = 4a_{n-1}-a_{n-2},\ \ n\geq 2.$$
Determine an odd prime divisor of $a_{2015}$.
\end{frame}

\begin{frame}{Nonlinear recursion}
Determine the value of the sequence $x_n$ satisfying $x_0 = 1$ and 
$$x_n = 1 + 1/x_{n-1},\ \ \text{for}\ \ n > 0$$
Doex $x_n$ converge?  If so, to what?
\end{frame}

\begin{frame}{Putnam 1993 A2}
Let $z_0$ and $z_1$ be real numbers.
Determine an explicit formula in terms of $z_0$ and $z_1$ for the value of
$$z_{n}^2-z_{n+1}z_{n-1} = 1,\ \ n \geq 1.$$
\end{frame}

\end{document}


